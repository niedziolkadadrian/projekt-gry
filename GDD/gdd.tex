\documentclass[12pt]{article}
\usepackage[utf8]{inputenc}
\usepackage{polski}

\author{Adrian Niedziółka-Domański}
\title {TBD\\
	\large Game Design Document}



\begin{document}
	\maketitle
	\pagebreak
	\tableofcontents
	\pagebreak
	
	\section{Wprowadzenie}
	Dokument ten ma na celu przybliżenie wyglądu gry i rozgrywki w niej prowadzonej. Ten dokument nie musi dokładnie odzwierciedlać finalnej postaci gry, gdyż wiele rzeczy może zmienić się w trakcie jej powstawania.
	
	\section{O grze}
	W tej grze zaczynamy jako pojedyńcza osoba, która wyrusza w świat, aby przeżywać swoją przygodę oraz pobudować swoje miejsce do życia. Gra toczy się w fantastycznym świecie gdzie mamy wiele do odkrycia, ale czyhają na nas także niebezpieczeństwa.
	
	\section{Fabuła}
	Zaczynamy w startowej miejscowości zwanej \textit{Startide}, z której nasza podróż w nieznane się zaczyna. Po drodze możemy spotkać różne osoby, które dadzą nam zadania do wykonania. Możemy pobudować swoje miejsce do życia w dowolnym miejscu na świecie, a następnie możemy zachęcić inne postacie, aby zamieszkały w naszej miejscowości, przy czym trzeba im także zapewnić wszystkie ich potrzeby, aby w niej zostały.
	
	\section{Świat}
	Świat gry jest światem fantasy, który jest archipelagiem wysp. Na świecie jest kilka miejscowości m.in.:
	\begin{itemize}
		\item \textit{Startide} - startowa miejscowość, mała, stąd pochodzi nasz bohater
		\item \textit{Hofdalur} - największa miejscowość. Jest głównym miastem na świecie. Jest centrum najważniejszych wydarzeń i marzeniem dla wielu ludzi z mniejszych miejscowości.
		\item \textit{Sillavere} - miejscowość górnicza i przemysłowa na tej samej wyspie co \textit{Hofdalur}
		\item \textit{Lillehus} - miejscowość rolnicza na tej samej wyspie co \textit{Hofdalur}
		\item Bonetide, Wildecoast, Madshire - średnie miejscowości rozsiane po świecie
		\item możliwe inne mniejsze miejscowości
	\end{itemize} 
	Świat ten zamieszkują także inne stworzenia.
	Niektóre zwierzęta/potwory mogą być agresywne i należy bronić się przed nimi, albo je omijać.
	\newline
	\newline
	Surowce:
	\begin{itemize}
		\item drewno - pozyskiwane z drzew
		\item żywność - pozyskiwane z farm, zwierząt itp.
		\item skóra - pozyskiwane ze zwierząt
		\item grimod - materiał organiczny znajdowany pod ziemią, szary, bardzo twardy, używany jako surowiec budowlany
		\item brimod - materiał organiczny znajdowany pod ziemią, srebrny, twardy, wykorzystywany do tworzenia broni
		\item dublony - waluta w grze
	\end{itemize}
	\section{Rozgrywka}
	Gracz ma swój ekwipunek (możliwe, że ograniczony wagowo). Oprócz tego pasek szybkiego wybory 1-9.\newline\newline
	Gracz ma także takie narzędzia jak np.:
	\begin{enumerate}
		\item młotek - pozwala on przejść w tryb budowy
		\item siekiera - pozwala ścinać drzewa
		\item kilof - pozwala wydobywać \textit{grimod} i \textit{brimod}
		\item miecz - pozwala na walkę z innymi stworzeniami/postaciami
		\item broń palna - pozwala na walkę na dystans
	\end{enumerate}
	Gracz może, rozmiawiać z niektórymi NPC i przyjmować od nich zadania np. na zabicie potwora/zwierząt, dostarczenie surowców oraz inne.\newline\newline
	Gracz może handlować z niektórymi NPC.
	\newline\newline
	Tryb budowy pozwala na ustawienie przezroczystego planu budynku. Aby go pobudować należy dostarczyć surowce.
	\newline\newline
	Crafting - gracz może tworzyć przedmioty z listy receptur. Wybiera on recepturę i z jego ekwipunku zostają zabrane surowce potrzebne do stworzenia danego przedmiotu i jest on tworzony.
	\section{Grafika}
	Grafika 3D, low poly. Perspektywa z 3 osoby z możliwością dostosowania odległości kamery, a także jej obrotu.
\end{document}