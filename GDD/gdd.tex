\documentclass[12pt]{article}
\usepackage[utf8]{inputenc}
\usepackage{polski}

\author{Adrian Niedziółka-Domański}
\title {GDD}


\begin{document}
	\maketitle
	\pagebreak
	\tableofcontents
	\pagebreak
	
	\section{Wprowadzenie}
	Witam we wprowadzeniu :D
	
	\section{O grze}
	Celem gry jest rozbudowa miejscowości, utrzymanie jego mieszkańców, a także wykoywanie zadań i budowa reputacji z róznymi frakcjami.
	\section{Fabuła}
	
	\section{Mechaniki}
	Zbieranie surowców t.j. drewno, kamień. Uprawa farm na żywność. Handel z innymi frakcjami, aby zdobyć materiały niedostępne w swojej miejscowości.
	
	Tryb budowania: Wybór budynku do postawienia. Przezroczysty model budynku, trzeba dostarczyć surowce, aby pobudować go. Możliwość budowania ścieżek, dróg oraz schodów.
	
	Handel: Ładowanie surowców na statek, przepłynięcie na inną wyspę i kupno/sprzedaż surowców tam.
	
	Ekwipunek gracza ograniczony wagowo. Każdy surowiec/przedmiot ma swoją wagę. Pasek szybkiego wyboru 1-9.
	\section{Świat}
	Światem jest archipelag wysp zamieszkanych przez różne frakcje. Jedna z wysp należy do gracza, gdzie buduje on swoją miejscowość.
	\section{Grafika}
	Grafika 3D low poly. Perspektywa z 3 osoby z możliwością dostosowania odległości kamery.
\end{document}